\documentclass[oneside,a4paper,11pt]{book}
\usepackage[utf8]{inputenc}
\usepackage{svg}
\usepackage[italian]{babel}
\usepackage{float}
\usepackage{fancyvrb}
\usepackage{titling}
\usepackage[margin=1in,footskip=0.25in]{geometry}
\usepackage{listings}
\usepackage[DIV=12,BCOR=2mm,headinclude=true,footinclude=false]{typearea}
\usepackage{color, colortbl,xcolor}
\usepackage[hidelinks]{hyperref}
\usepackage{tcolorbox}
\usepackage{chngcntr}
\usepackage{diagbox}
\usepackage{calc}
\usepackage{amssymb}
\usepackage{subcaption}
\usepackage{amsthm}
\usepackage{amsfonts}
\usepackage{mathtools}
\usepackage{parskip}
\usepackage{cancel}
\usepackage{forest}
\usepackage{listings}
\usepackage{mathrsfs}
\usepackage{enumitem}
\usepackage{makecell}
\usepackage{tikz}
\usepackage{pgfplots}
\pgfplotsset{compat=1.18}
\usepackage{fancyhdr}
\fancypagestyle{plain}{\fancyhf{}\renewcommand{\headrulewidth}{0pt}}
\pagestyle{fancy}
\fancyhf{}% Clear header/footer
\fancyhead[L]{\nouppercase\leftmark}
\fancyhead[R]{\thepage}
\usetikzlibrary{positioning,shapes.geometric,arrows.meta,matrix,automata,decorations.pathmorphing,patterns,decorations.pathreplacing,shapes.multipart,calc,snakes}
\usetikzlibrary{arrows.meta, backgrounds, chains, positioning, shapes.geometric, shapes.multipart}
\tcbuselibrary{skins}
\counterwithin{figure}{section}
%Nuovi comandi
\newcommand\myeq{\stackrel{\mathclap{\normalfont\mbox{def}}}{=}}
\newcommand\prodG{\stackrel{\mathclap{\normalfont\mbox{\tiny{G}}}}{\Longrightarrow}}
%asmthm
\newlength{\marginlabelsep}\setlength{\marginlabelsep}{0.5em}
\newtheoremstyle{italicstyle} %% Name
  {} %% <- Space above (empty = default = \topsep = 8.0pt plus 2.0pt minus 4.0pt)
  {} %% <- Space below (empty = default = \topsep = 8.0pt plus 2.0pt minus 4.0pt)
  {\itshape} %% <- Body font
  {} %% <- Indent amount (empty = no indent, \parindent = just that)
  {\bfseries} %% <- Thm head font
  {} %% <- Punctuation after thm head
  {1pt} %% <- Space after thm head (or " " or \newline) (default: 5pt plus 1pt minus 1pt)
  {\vtop to 0pt{\llap{\thmname{#1}\hskip\marginlabelsep}
                \llap{\thmnumber{#2}\hskip\marginlabelsep}}\thmnote{#3\\}%
  }
\newtheoremstyle{normStyle} %% Name
  {} %% <- Space above (empty = default = \topsep = 8.0pt plus 2.0pt minus 4.0pt)
  {} %% <- Space below (empty = default = \topsep = 8.0pt plus 2.0pt minus 4.0pt)
  {\normalfont} %% <- Body font
  {} %% <- Indent amount (empty = no indent, \parindent = just that)
  {\bfseries} %% <- Thm head font
  {} %% <- Punctuation after thm head
  {1pt} %% <- Space after thm head (or " " or \newline) (default: 5pt plus 1pt minus 1pt)
  {\vtop to 0pt{\llap{\thmname{#1}\hskip\marginlabelsep}
                \llap{\thmnumber{#2}\hskip\marginlabelsep}}\thmnote{#3\\}%
  }
\theoremstyle{italicstyle}
\newtheorem{corollary}{Corollario}[section]
\newtheorem{notazione}{Notazione}[section]
\newtheorem{lemma}{Lemma}[section]
\newtheorem{definizione}{Definizione}[section]
\newtheorem{nota}{Nota}[section]
\newtheorem{exercise}{Esercizio}[section]
\theoremstyle{normStyle}
\newtheorem{exmp}{Esempio}[section]
\newtheorem{theorem}{Teorema}[section]
\newtheorem{proposizione}{Proposizione}[section]
\tcbuselibrary{listings,skins}
\newtcblisting{mylisting}[2][]{
    arc=0pt, outer arc=0pt,
    listing only, 
    title=#2,
    #1,
    listing options= {escapechar=|}
}
\newcommand{\myboxedtext}[2][rectangle,draw]{%
    \tikz[baseline=-0.6ex] \node [#1]{#2};}%
%%======================================================================
\title{Capture the Flag Manual}
\author{\textit{Alessio Gjergji}\\
\textit{Author 2} \\
\textit{Author 3} \\
\textit{Author 4} \\
\textit{Author 5}}
\date{}
\begin{document}
\maketitle
\tableofcontents
\chapter{Comandi terminale}
\section{Introduzione}
Durante una ctf potremmo trovarci di fronte ad alcune challenge in cui è necessario l'utilizzo di vari comandi della shell per recuperare la flag richiesta e quindi passare alla challenge successiva.
Di seguito vedremo alcuni comandi della shell per sistemi UNIX o macOS che possono tornare utili.

\section{Comandi di base}
Se si conosce un comando ma non si sa come utilizzarlo è bene consultare il manuale scrivendo sul terminale \textbf{man} $\langle command \rangle$. Se non è presente la pagina del manuale provare a specificare il flag \textbf{--help}. 

\section{Operazioni sulle directory}
\subsection{cd}
Per navigare attraverso il filesystem utilizziamo il comando \textbf{cd dir}.
\begin{itemize}
    \item \textbf{cd esempio} (ci spostiamo nella cartella \textit{esempio})
    \item \textbf{cd Dekstop/esempio} (ci spostiamo nella cartella \textit{esempio} identificata dal suo path)
    \item \textbf{cd ..} (ci permette di spostarci nella cartella superiore)
    \item \textbf{cd $\sim$} (ci spostiamo nella home directory)
    \item \textbf{cd /} (ci spostiamo nella root directory)
\end{itemize}
Per creare una cartella utilizziamo il comando \textbf{mk dir}, se invece vogliamo vedere la cartella corrente utlizziamo il comando \textbf{pwd}.

\subsection{ls}
Se vogliamo vedere i file all'interno di una cartella utilizziamo il comando \textbf{ls}.
\begin{itemize}
    \item \textbf{ls} (mostra i file nella cartella corrente)
    \item \textbf{ls Desktop/esempio} (mostra i file all'interno della cartella \textit{esempio} identificata dal path)
\end{itemize}
Tra le opzioni del comando \textbf{ls} possiamo trovare:
\begin{itemize}
    \item \textbf{-a} (mostra tutti i file, inclusi quelli nascosti)
    \item \textbf{-r} (inverte l'ordine della lista)
    \item \textbf{-t} (ordina in base all'ultimo modificato)
    \item \textbf{-S} (ordina per dimensione del file)
\end{itemize}

\section{Operazioni sui file}
\subsection{touch}
Per creare un file utilizzare il comando \textbf{touch file}.

\subsection{cat}
Il comando \textbf{cat} permette di concatenare file e stampare il loro contenuto sullo standard output.
\begin{itemize}
    \item \textbf{cat file} (stampa il contenuto del file, se vengono specificati più file li concatena e stampa il contenuto, e.g. cat file file2)
    \item \textbf{cat $<$ -file} (permette di stampare il contenuto di un file con il nome che inizia con un dash)
    \item \textbf{cat $"nome file con spazi"$} (permette di stampare il contenuto di un file che contiene spazi nel nome)
    \item \textbf{cat .file} (stampa il contenuto del file nascosto)
\end{itemize}

\subsection{cp, mv, rm, file}
\begin{itemize}
    \item \textbf{cp file file2} (copia file in file2)
    \item \textbf{mv file file2} (rinomino file in file2)
    \item \textbf{rm file} (elimino file)
        \begin{itemize}
            \item \textbf{rm -r} (rimuovo le directory e i loro contenuti)
            \item \textbf{rm -d} (rimuovo direcotry vuote)
        \end{itemize}
    \item \textbf{file file1} (ritorna il tipo di file1)
\end{itemize}

\subsection{head, tail}
\begin{itemize}
    \item \textbf{head file1} (ritorna le prime 10 linee di file1)
    \item \textbf{tail file1} (ritorna le ultime 10 linee di file1)
\end{itemize}

\subsection{strings}
Il comando \textbf{strings} stampa una sequenza di stringhe leggibili all'interno di un file.
\begin{itemize}
    \item \textbf{strings file}
    \begin{itemize}
        \item con il flag \textbf{-n number-of-lines} (specifichiamo la lunghezza minima delle stringhe)
        \item con il flag \textbf{-e encoding} (specifichiamo la codifica)
        \item con il flag \textbf{-w} (includiamo gli spazi bianchi)
        \item con il flag \textbf{-s} (il separatore per l'output)
    \end{itemize}
\end{itemize}

\section{Ricerca}
\subsection{sort, unique}
Il comando \textbf{sort file} permette di ordinare le linee all'interno di un file, il comando \textbf{unique -u} permette di mostrare le linee uniche non duplicate, questi due comandi possono essere comodi da usare in combinazione attraverso l'utilizzo di una pipe: \textbf{sort nomefile | unique -u}.

\subsection{grep}
Il comando \textbf{grep pattern files} permette di cercare un determinato pattern in ogni file, i pattern andrebbero specificati sempre compresi tra doppi apici.

\begin{itemize}
    \item \textbf{grep -i } (ricerca case-insensitive)
    \item \textbf{grep -r } (ricerca ricorsiva)
    \item \textbf{grep -v } (ricerca invertita)
    \item \textbf{grep -o } (mostra solo la parte di file che ha matchato il pattern)
\end{itemize}

\subsection{find}
Il comando \textbf{find} permette di cercare dei file all'interno del filesystem.

\begin{itemize}
    \item \textbf{find /percorso -name "filename"} (ricerca per nome)
    \item \textbf{find /percorso -name "*.txt"} (ricerca per estensione)
    \item \textbf{find /percorso -type f -size +1M} (ricerca per dimensione)
    \item \textbf{find /percorso -user utente -group gruppo} (ricerca per proprietario e gruppo)
    \item \textbf{find /percorso -mtime -7} (ricerca per data di modifica)
\end{itemize}

\section{Connessioni ssh o tcp}
Connessione ad una risorsa in ssh:

\begin{itemize}
    \item \textbf{ssh -p numero-porta utente@indirizzo-del-server}
\end{itemize}

Connessione tramite tcp:

\begin{itemize}
    \item \textbf{nc host port}
\end{itemize}

\section{Gestione dei processi}
\begin{itemize}
    \item \textbf{ps} (mostra uno snapshot dei processi)
    \item \textbf{top} (mostra i processi real-time)\item \textbf{}
    \item \textbf{kill pid} (termina un processo con il pid=pid)
    \item \textbf{pkill name} (termina un processo col nome=name)
    \item \textbf{killall name} (termina tutti i processi con il nome che inizia per name)
\end{itemize}

\end{document}
