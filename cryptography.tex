\chapter{Crittografia}
\section{Conversioni}
\subsection{Decodifica da numero in base 2 a big endian}
\begin{lstlisting}[style=pythonStyle]
    result = '101010'
    result = int(result, 2).to_bytes((result.bit_length() + 7) // 8, 'big').decode('utf-8')
\end{lstlisting}
\subsection{Decodifica da numero in base 2 a little endian}
\begin{lstlisting}[style=pythonStyle]
    result = '101010'
    result = int(result, 2).to_bytes((result.bit_length() + 7) // 8, 'little').decode('utf-8')
\end{lstlisting}
\subsection{Decodifica da numero in base 8 a big endian}
\begin{lstlisting}[style=pythonStyle]
    result = '1234'
    result = int(result, 8).to_bytes((result.bit_length() + 7) // 8, 'big').decode('utf-8')
\end{lstlisting}
\subsection{Decodifica da numero in base 8 a little endian}
\begin{lstlisting}[style=pythonStyle]
    result = '1234'
    result = int(result, 8).to_bytes((result.bit_length() + 7) // 8, 'little').decode('utf-8')
\end{lstlisting}
\subsection{Decodifica da numero in base 10 a big endian}
\begin{lstlisting}[style=pythonStyle]
    result = 1234
    result = result.to_bytes((result.bit_length() + 7) // 8, 'big').decode('utf-8')
\end{lstlisting}
\subsection{Decodifica da numero in base 10 a little endian}
\begin{lstlisting}[style=pythonStyle]
    result = 1234
    result = result.to_bytes((result.bit_length() + 7) // 8, 'little').decode('utf-8')
\end{lstlisting}
\subsection{Decodifica da numero in base 16 a big endian}
\begin{lstlisting}[style=pythonStyle]
    result = '1234'
    result = bytes.fromhex(result).decode('utf-8')
\end{lstlisting}
\subsection{Decodifica da numero in base 16 a little endian}
\begin{lstlisting}[style=pythonStyle]
    result = '1234'
    result = bytes.fromhex(result)[::-1].decode('utf-8')
\end{lstlisting}
\subsection{Decodifica da numero in base 32 a big endian}
\begin{lstlisting}[style=pythonStyle]
    import base64

    result = '1234'
    result = base64.b32decode(result).decode('utf-8')
\end{lstlisting}
\subsection{Decodifica da numero in base 32 a little endian}
\begin{lstlisting}[style=pythonStyle]
    import base64

    result = '1234'
    result = base64.b32decode(result)[::-1].decode('utf-8')
\end{lstlisting}
\subsection{Decodifica da numero in base 58 a big endian}
\begin{lstlisting}[style=pythonStyle]
    import base58

    result = '1234'
    result = base58.b58decode(result).decode('utf-8')
\end{lstlisting}
\subsection{Decodifica da numero in base 58 a little endian}
\begin{lstlisting}[style=pythonStyle]
    import base58

    result = '1234'
    result = base58.b58decode(result)[::-1].decode('utf-8')
\end{lstlisting}
\subsection{Decodifica da numero in base 64 a big endian}
\begin{lstlisting}[style=pythonStyle]
    import base64

    result = '1234'
    result = base64.b64decode(result).decode('utf-8')
\end{lstlisting}
\subsection{Decodifica da numero in base 64 a little endian}
\begin{lstlisting}[style=pythonStyle]
    import base64

    result = '1234'
    result = base64.b64decode(result)[::-1].decode('utf-8')
\end{lstlisting}
\section{One-Time Pad}
Un dettaglio fondamentale per ottenere sicurezza perfetta è che la chiave sia lunga tanto quanto il messaggio.

Nel caso in cui la chiave sia molto corta (\textit{e ripetuta, per esempio})
potrebbe essere possibile un attacco a forza bruta: provare tutte le chiavi
candidate e vedere per quale si ottiene un risultato sensato.
La ripetizione di pattern all'interno del messaggio cifrato è un indizio
che la chiave sia corta.
\begin{lstlisting}[style=pythonStyle]
    def xor(a, b):
        return bytes([x^y for x,y in zip(a,b)])

    num = bytes.fromhex('104e137f42')

    for i in range(256):
        key = i.to_bytes(1, 'big')*len(num)
        result = xor(num, key)
        if result.isascii():
            result = result.decode('ascii')
            print('key: ', i, 'result: ', result)

\end{lstlisting}
Nel caso descritto la chiave è di un solo byte, quindi si prova a cifrare
il messaggio con tutte le chiavi possibili e si controlla se il risultato
è un testo \texttt{ASCII}.
\section{Crittoanalisi differenziale}
Per utilizzare la crittoanalisi differenziale è necessario avere a disposizione
un \textit{oracolo} che cifri un messaggio arbitrario con una chiave segreta.
Installando il tool \texttt{MTP} possiamo ottenere una parziale implementazione
del messaggio.

\begin{lstlisting}[style=bashStyle]
    $ pip3 install mtp
    $ mtp <file>
\end{lstlisting}
\section{Libreria \texttt{PyCryptodome}}
\subsection{\texttt{DES}}
\begin{lstlisting}[style=pythonStyle]
    from Crypto.Cipher import DES
    from Crypto.Util.Padding import pad
    from Crypto.Random import get_random_bytes
    from binascii import hexlify, unhexlify

    # Chiave in formato esadecimale
    key_hex = '1bb4545801ca1e93'
    key = unhexlify(key_hex)

    # Testo in chiaro
    plaintext = 'La lunghezza di questa frase non e divisibile per 8'

    # Applicare lo schema di riempimento x923
    plaintext = pad(plaintext.encode('utf-8'), DES.block_size, style='x923')

    # Generare un vettore di inizializzazione casuale
    iv = get_random_bytes(DES.block_size)

    # stampo Che IV hai utilizzato (in esadecimale)
    print('IV: ' + hexlify(iv).decode('utf-8') + '\n')

    # Creare un oggetto DES in modalita' CBC
    cipher = DES.new(key, DES.MODE_CBC, iv)

    # Cifrare il testo in chiaro
    ciphertext = cipher.encrypt(plaintext)

    # Stampare il testo cifrato in formato esadecimale
    print("Text: " + hexlify(ciphertext).decode('utf-8') + "\n")
\end{lstlisting}

\subsection{\texttt{AES}}

\begin{lstlisting}[style=pythonStyle]
    from Crypto.Cipher import AES
    from Crypto.Util.Padding import pad
    from Crypto.Random import get_random_bytes
    from binascii import hexlify, unhexlify

    # Generare una chiave casuale AES256 in formato esadecimale
    key = get_random_bytes(32)  # 32 byte per AES256

    # Testo in chiaro
    plaintext = 'Mi chiedo cosa significhi il numero nel nome di questo algoritmo.'

    # Applicare lo schema di riempimento PKCS7
    plaintext = pad(plaintext.encode('utf-8'), AES.block_size)

    # Generare un vettore di inizializzazione casuale
    iv = get_random_bytes(AES.block_size)

    # Stampo che IV hai utilizzato (in esadecimale)
    print("IV:", hexlify(iv).decode('utf-8'))

    # Creare un oggetto AES in modalita' CFB con segment size di 24
    cipher = AES.new(key, AES.MODE_CFB, iv, segment_size=24)

    # Cifrare il testo in chiaro
    ciphertext = cipher.encrypt(plaintext)

    # Stampare la chiave in formato esadecimale
    print("Chiave:", hexlify(key).decode('utf-8'))

    # Stampare il testo cifrato in formato esadecimale
    print("Testo cifrato:", hexlify(ciphertext).decode('utf-8'))
\end{lstlisting}
\subsection{Stream cipher}
\begin{lstlisting}[style = pythonStyle]
    from Crypto.Cipher import ChaCha20
    from binascii import unhexlify

    # Chiave in formato esadecimale
    key_hex = '9d6a8f...'
    key = unhexlify(key_hex)

    # Testo cifrato in formato esadecimale
    ciphertext_hex = 'b7e9ac6...'
    ciphertext = unhexlify(ciphertext_hex)

    # Nonce in formato esadecimale
    nonce_hex = '76c24201d...'
    nonce = unhexlify(nonce_hex)

    # Creare un oggetto ChaCha20
    cipher = ChaCha20.new(key=key, nonce=nonce)

    # Decifrare il testo cifrato
    plaintext = cipher.decrypt(ciphertext)

    # Stampa il testo in chiaro
    print(plaintext.decode('utf-8'))
\end{lstlisting}
\subsection{Hash}
\begin{lstlisting}[style=pythonStyle]
    from Crypto.Hash import SHA256
    from binascii import hexlify

    # Testo in chiaro
    plaintext = 'La lunghezza di questa frase non e divisibile per 8'

    # Creare un oggetto SHA256
    h = SHA256.new()

    # Aggiornare l'hash con il testo in chiaro
    h.update(plaintext.encode('utf-8'))

    # Calcolare l'hash
    digest = h.digest()

    # Stampa l'hash in formato esadecimale
    print(hexlify(digest).decode('utf-8'))
\end{lstlisting}
\section{Aritmetica Modulare}

Diciamo che $a \equiv b \pmod{n}$ se $n$ divide $a-b$.

\begin{itemize}
    \item $a \equiv b \pmod{n} \iff a \bmod{n} = b \bmod{n}$
    \item $a \equiv b \pmod{n} \iff \exists k \in \mathbb{Z} \mid a = b + kn$
    \item $a \equiv b \pmod{n} \iff \exists k \in \mathbb{Z} \mid a \bmod{n} = b + kn$
\end{itemize}

\subsection{Proprietà}

\begin{itemize}
    \item $a \equiv a \pmod{n}$
    \item $a \equiv b \pmod{n} \implies b \equiv a \pmod{n}$
    \item $a \equiv b \pmod{n} \land b \equiv c \pmod{n} \implies a \equiv c \pmod{n}$
    \item $a \equiv b \pmod{n} \land c \equiv d \pmod{n} \implies a + c \equiv b + d \pmod{n}$
    \item $a \equiv b \pmod{n} \land c \equiv d \pmod{n} \implies ac \equiv bd \pmod{n}$
\end{itemize}

\subsection{Aritmetica Modulare}

La \emph{aritmetica modulare} è una branca dell'aritmetica che si
occupa delle operazioni sui numeri interi modulari rispetto a un modulo
dato. Un concetto fondamentale è il \emph{resto} di una divisione, spesso indicato come $a \mod m$, che rappresenta
il residuo della divisione di $a$ per $m$.

Un'applicazione comune dell'aritmetica modulare è nel calcolo del \emph{mcd} (massimo comune divisore) di due numeri.
La funzione di Euclide è una tecnica classica che sfrutta l'aritmetica modulare per calcolare l'mcd. Ad esempio,
se abbiamo due numeri $a$ e $b$, possiamo utilizzare la funzione di Euclide estesa per trovare i coefficienti di
Bezout $x$ e $y$ tali che $a \cdot x + b \cdot y = \texttt{mcd}(a, b)$.

\begin{equation}
\texttt{gcd}(a, b) = ax + by
\end{equation}

In Python creiamo una funzione che calcola l'mcd di due numeri utilizzando la funzione di Euclide estesa.

\begin{lstlisting}[style=pythonStyle]
def gcd(a, b):
    if a == 0:
        return b, 0, 1
    else:
        g, y, x = gcd(b % a, a)
        return g, x - (b // a) * y, y
\end{lstlisting}

Questo concetto è spesso applicato in crittografia, dove la sicurezza si basa su operazioni modulari e il teorema di Fermat.
